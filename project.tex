\documentclass[a4paper]{article}

%% Language and font encodings
\usepackage[english]{babel}
\usepackage[utf8]{inputenc}
\usepackage[T1]{fontenc}
\usepackage{verbatim}
%% Sets page size and margins
\usepackage[a4paper,top=3cm,bottom=2cm,left=3cm,right=3cm,marginparwidth=1.75cm]{geometry}

%% Useful packages
\usepackage{amsmath}
\usepackage{graphicx}
\usepackage[colorinlistoftodos]{todonotes}
\usepackage[colorlinks=true, allcolors=blue]{hyperref}
%\usepackage{natbib}
\usepackage[natbibapa]{apacite}
\title{Some Experimentations with Dynamic Stochastic General Equilibrium Models}
\author{Nelson Seixas dos Santos \\ Bernardo Hilleshein Paulsen \\ Gabriel Roman \\ Guilherme Luft Mendes \\ Universidade Federal do Rio Grande do Sul\\ Faculdade de Ciências Econômicas \\ Departamento de Economia e Relações Internacionais}

\begin{document}
	\maketitle
	
	
	
	\section{The Research Problem}
	
	The problem of simulating an economy has long been studied in the literature.  A very good introduction to this issue may be found in \citet{mccandless2008} 
	
	\section{The Importance of the problem}
	
	\section{The dynamics of Brazilian economy}
	Y, C, I, G, T, M, i
	
	\section{Dynamic Stochastic General Equilibrium Models}
	
	
	
	\section{Experiments, Methods and Data}
	
	
	
	\section{Expected Results and Discussion}
	
	
	\section{Timetable}
	
	
	\section{Conclusion}
	
	
	
	\begin{comment}
	
	
	\begin{figure}
	\centering
	\includegraphics[width=0.3\textwidth]{frog.jpg}
	\caption{\label{fig:frog}This frog was uploaded via the project menu.}
	\end{figure}
	
	\subsection{How to add Comments}
	
	Comments can be added to your project by clicking on the comment icon in the toolbar above. % * <john.hammersley@gmail.com> 2016-07-03T09:54:16.211Z:
	%
	% Here's an example comment!
	%
	To reply to a comment, simply click the reply button in the lower right corner of the comment, and you can close them when you're done.
	
	Comments can also be added to the margins of the compiled PDF using the todo command\todo{Here's a comment in the margin!}, as shown in the example on the right. You can also add inline comments:
	
	\todo[inline, color=green!40]{This is an inline comment.}
	
	\subsection{How to add Tables}
	
	Use the table and tabular commands for basic tables --- see Table~\ref{tab:widgets}, for example. 
	
	\begin{table}
	\centering
	\begin{tabular}{l|r}
	Item & Quantity \\\hline
	Widgets & 42 \\
	Gadgets & 13
	\end{tabular}
	\caption{\label{tab:widgets}An example table.}
	\end{table}
	
	
	
	\subsection{How to add Citations and a References List}
	
	You can upload a \verb|.bib| file containing your BibTeX entries, created with JabRef; or import your \href{https://www.overleaf.com/blog/184}{Mendeley}, CiteULike or Zotero library as a \verb|.bib| file. You can then cite entries from it, like this: \cite{greenwade93}. Just remember to specify a bibliography style, as well as the filename of the \verb|.bib|.
	
	You can find a \href{https://www.overleaf.com/help/97-how-to-include-a-bibliography-using-bibtex}{video tutorial here} to learn more about BibTeX.
	
	We hope you find Overleaf useful, and please let us know if you have any feedback using the help menu above --- or use the contact form at \url{https://www.overleaf.com/contact}!
	\end{comment}
	\bibliographystyle{apacite}
	\bibliography{dsge}
	
\end{document}